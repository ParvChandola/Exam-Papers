\documentclass[journal,12pt,twocolumn]{IEEEtran}
%
\usepackage{setspace}
\usepackage{gensymb}
\usepackage{tabularx}
%\doublespacing
\singlespacing
\usepackage{float}
%\usepackage{graphicx}
%\usepackage{amssymb}
%\usepackage{relsize}
\usepackage[cmex10]{amsmath}
%\usepackage{amsthm}
%\interdisplaylinepenalty=2500
%\savesymbol{iint}
%\usepackage{txfonts}
%\restoresymbol{TXF}{iint}
%\usepackage{wasysym}
\usepackage{amsthm}
%\usepackage{iithtlc}
\usepackage{mathrsfs}
\usepackage{txfonts}
\usepackage{stfloats}
\usepackage{bm}
\usepackage{cite}
\usepackage{cases}
\usepackage{subfig}
%\usepackage{xtab}
\usepackage{longtable}
\usepackage{multirow}
%\usepackage{algorithm}
%\usepackage{algpseudocode}
\usepackage{enumitem}
\usepackage{mathtools}
\usepackage{tikz}
\usepackage{circuitikz}
\usepackage{verbatim}
%\usepackage{tfrupee}
\usepackage[breaklinks=true]{hyperref}
%\usepackage{stmaryrd}
\usepackage{tkz-euclide} % loads  TikZ and tkz-base
%\usetkzobj{all}
\usepackage{listings}
    \usepackage{color}                                            %%
    \usepackage{array}                                            %%
    \usepackage{longtable}                                        %%
    \usepackage{calc}                                             %%
    \usepackage{multirow}                                         %%
    \usepackage{hhline}                                           %%
    \usepackage{ifthen}                                           %%
  %optionally (for landscape tables embedded in another document): %%
    \usepackage{lscape}     
\usepackage{multicol}
\usepackage{chngcntr}
\def\inputGnumericTable{}                                
%\usepackage{enumerate}

%\usepackage{wasysym}
%\newcounter{MYtempeqncnt}
\DeclareMathOperator*{\Res}{Res}
%\renewcommand{\baselinestretch}{2}
\renewcommand\thesection{\arabic{section}}
\renewcommand\thesubsection{\thesection.\arabic{subsection}}
\renewcommand\thesubsubsection{\thesubsection.\arabic{subsubsection}}

\renewcommand\thesectiondis{\arabic{section}}
\renewcommand\thesubsectiondis{\thesectiondis.\arabic{subsection}}
\renewcommand\thesubsubsectiondis{\thesubsectiondis.\arabic{subsubsection}}

% correct bad hyphenation here
\hyphenation{op-tical net-works semi-conduc-tor}
\def\inputGnumericTable{}                                 %%

\lstset{
%language=C,
frame=single, 
breaklines=true,
columns=fullflexible
}
%\lstset{
%language=tex,
%frame=single, 
%breaklines=true
%}

\begin{document}
%


\newtheorem{theorem}{Theorem}[section]
\newtheorem{problem}{Problem}
\newtheorem{proposition}{Proposition}[section]
\newtheorem{lemma}{Lemma}[section]
\newtheorem{corollary}[theorem]{Corollary}
\newtheorem{example}{Example}[section]
\newtheorem{definition}[problem]{Definition}
%\newtheorem{thm}{Theorem}[section] 
%\newtheorem{defn}[thm]{Definition}
%\newtheorem{algorithm}{Algorithm}[section]
%\newtheorem{cor}{Corollary}
\newcommand{\BEQA}{\begin{eqnarray}}
\newcommand{\EEQA}{\end{eqnarray}}
\newcommand{\define}{\stackrel{\triangle}{=}}

\bibliographystyle{IEEEtran}
%\bibliographystyle{ieeetr}


\providecommand{\mbf}{\mathbf}
\providecommand{\pr}[1]{\ensuremath{\Pr\left(#1\right)}}
\providecommand{\qfunc}[1]{\ensuremath{Q\left(#1\right)}}
\providecommand{\sbrak}[1]{\ensuremath{{}\left[#1\right]}}
\providecommand{\lsbrak}[1]{\ensuremath{{}\left[#1\right.}}
\providecommand{\rsbrak}[1]{\ensuremath{{}\left.#1\right]}}
\providecommand{\brak}[1]{\ensuremath{\left(#1\right)}}
\providecommand{\lbrak}[1]{\ensuremath{\left(#1\right.}}
\providecommand{\rbrak}[1]{\ensuremath{\left.#1\right)}}
\providecommand{\cbrak}[1]{\ensuremath{\left\{#1\right\}}}
\providecommand{\lcbrak}[1]{\ensuremath{\left\{#1\right.}}
\providecommand{\rcbrak}[1]{\ensuremath{\left.#1\right\}}}
\theoremstyle{remark}
\newtheorem{rem}{Remark}
\newcommand{\sgn}{\mathop{\mathrm{sgn}}}
\providecommand{\abs}[1]{\left\vert#1\right\vert}
\providecommand{\res}[1]{\Res\displaylimits_{#1}} 
\providecommand{\norm}[1]{\left\lVert#1\right\rVert}
%\providecommand{\norm}[1]{\lVert#1\rVert}
\providecommand{\mtx}[1]{\mathbf{#1}}
\providecommand{\mean}[1]{E\left[ #1 \right]}
\providecommand{\fourier}{\overset{\mathcal{F}}{ \rightleftharpoons}}
%\providecommand{\hilbert}{\overset{\mathcal{H}}{ \rightleftharpoons}}
\providecommand{\system}{\overset{\mathcal{H}}{ \longleftrightarrow}}
	%\newcommand{\solution}[2]{\textbf{Solution:}{#1}}
\newcommand{\solution}{\noindent \textbf{Solution: }}
\newcommand{\cosec}{\,\text{cosec}\,}
\providecommand{\dec}[2]{\ensuremath{\overset{#1}{\underset{#2}{\gtrless}}}}
\newcommand{\myvec}[1]{\ensuremath{\begin{pmatrix}#1\end{pmatrix}}}
\newcommand{\mydet}[1]{\ensuremath{\begin{vmatrix}#1\end{vmatrix}}}
%\numberwithin{equation}{section}
%\numberwithin{equation}{subsection}
%\numberwithin{problem}{section}
%\numberwithin{definition}{section}
\makeatletter
\@addtoreset{figure}{problem}
\makeatother

\let\StandardTheFigure\thefigure
\let\vec\mathbf
%\renewcommand{\thefigure}{\theproblem.\arabic{figure}}
\renewcommand{\thefigure}{\theproblem}
%\setlist[enumerate,1]{before=\renewcommand\theequation{\theenumi.\arabic{equation}}
%\counterwithin{equation}{enumi}


%\renewcommand{\theequation}{\arabic{subsection}.\arabic{equation}}

\def\putbox#1#2#3{\makebox[0in][l]{\makebox[#1][l]{}\raisebox{\baselineskip}[0in][0in]{\raisebox{#2}[0in][0in]{#3}}}}
     \def\rightbox#1{\makebox[0in][r]{#1}}
     \def\centbox#1{\makebox[0in]{#1}}
     \def\topbox#1{\raisebox{-\baselineskip}[0in][0in]{#1}}
     \def\midbox#1{\raisebox{-0.5\baselineskip}[0in][0in]{#1}}

\title{MATHEMATICS\\ \normalsize CBSE class X}
\maketitle
%\tableofcontents
\begin{abstract}

     \textbf{General instructions : }
     
     \begin{enumerate}
     \item \textit{All questions are compulsory.}
     
     \item \textit{The question paper consists of 29 questions divided into four sections A, B, C and D. Section A comprises of 4 questions of one mark each, Section B comprises of 8 questions of two marks each, Section C comprises of 11 questions of four marks each and Section D comprises of 6 questions of six marks each.}
     
     \item \textit{All questions in Section A are to be answered in one word, one sentence or as per the exact requirement of the question.}
     
     \item \textit{There is no overall choice. However, internal choice has been provided in 3 questions of four marks each and 3 questions of six marks each. You have to attempt only one of the alternatives in all such questions.}
     
     \item \textit{Use of calculators is not permitted. You may ask for logarithmic tables, if required.}
     \end{enumerate}
     
\end{abstract}
\begin{center}
     \section*{Section A}
     \end{center}
     
     \begin{flushleft}
      \textit{ Question numbers 1 to 4 carry 1 mark each.}
     \end{flushleft}
    
     
     
     \vspace{4mm}
    
     \begin{enumerate}
     
     \item Find the value of $ \tan^{-1}\sqrt{3} - \cot^{-1}\sqrt{-3} $
     
     \item \noindent If the matrix A = 
     $\begin{bmatrix}
     0 & a & -3 \\
     2 & 0 & -1 \\
     b & 1 & 0 \\
     \end{bmatrix} $
     is skew symmetric, find the values of ‘a’ and      ‘b’.
     
     \item Find the magnitude of each of the two vectors $\vec{a}$ and $\vec{b}$ having the same magnitude such that the angle between them is $60\degree$ and there scalar product is $\frac{9}{2}$
     
     \item If $ a \circledast b $ denotes the larger od 'a' and 'b' and if $ a \circ b = (a \circledast b) + 3 $ then the value of $ (5 \circ 10 ) $, where $\circledast $ and $ \circ $ are binary operations.
     
    \begin{center}
    \section*{Section B}
    \end{center}
    
    \begin{flushleft}
      \textit{ Question numbers 5 to 12 carry 2 mark each.}
     \end{flushleft}
    
    
    \item Prove that 
    $$ 3\sin^{-1}(x) = \sin^{-1}(3x-4x^{3}),\hspace{4mm} x \epsilon \hspace{2mm} [\frac{-1}{2},\frac{1}{2}] $$
    
    \item \noindent Given A = 
 $\begin{bmatrix}
  2 & -3\\ 
  4 & 7
\end{bmatrix}$,  
,compute $A^{-1}$ and show that $2A^{-1} = 9I - A $

    \item Differentiate
    $ \tan^{-1}(\frac{1 + \cos x}{\sin x})$
    
    \item The total cost C(x) associated with the production of x units of an item is given by C(x) = 
$ 0.005x^{3} - 0.02x^{2} + 30x + 5000 $. Find the marginal cost when 3 units are produced, where by marginal cost we mean the instantaneous rate of change of total cost at any level of output.

    \item Evaluate : 
    $$ \int_{}^{} \frac{\cos 2x + 2\sin^{2} x}{\cos^{2} x} \,dx $$
    
    \item Find the differential equation representing the family of curves $ y = ae^{bx + 5} $ , where a and b are arbitrary constants.
    
    \item If $\theta$ is the angle between two vectors $\hat{i} - 2\hat{j} + 3\hat{k}$ and $3\hat{i} - 2\hat{j} + \hat{k}$, find $\sin\theta$.
    
    \item A black and a red die are rolled together. Find the conditional probability of obtaining the sum 8, given that the red die resulted in a number less than 4.
    
    \begin{center}
    \section*{Section C}
    \end{center}
    
    \begin{flushleft}
    \textit{Question numbers 13 to 23 carry 4 marks each.}
    \end{flushleft}
    
    \item \noindent Using properties of determinant prove that
     $$\begin{vmatrix}
     1 & 1 & 1 + 3x \\
     1 + 3y & 1 & 1 \\
     1 & 1 + 3z & 1 \\
     \end{vmatrix} 
     = 9(3xyz + xy + yz + zx)$$.
     
     \item If 
     $ (x^{2} + y^{2})^{2} = xy $, find $ \frac{dy}{dx}$
     
     \begin{center}
     \textbf{OR}
     \end{center}
     
     If $ x = a(2\theta - \sin 2\theta )$ and 
     $ y = a(1 - \cos 2\theta)$, find $\frac{dy}{dx}$ when $\theta = \frac{\pi}{3}$.
     
     \item If $ y = \sin (\sin x) $,
     prove that 
     $ \frac{d^{2}y}{dx^{2}} + \tan x\frac{dy}{dx} + y\cos^{2} x = 0 $
     
     \item Find the equations of the tangent and the normal, to the curve $ 16x^{2} + 9y^{2} = 145 $ at the point $(x_{1} , y_{1}$ where $x_{1} = 2$ 
     and $y_{1} > 0$.
     
     \item An open tank with a square base and vertical sides is to be constructed from a metal sheet so as to hold a given quantity of water. Show that the cost of material will be least when depth of the tank is half of its width. If the cost is to be borne by nearby settled lower income families, for whom water will be provided, what kind of value is hidden in this question ?
     
     \item Find :
     $$ \int_{}^{} \frac{2\cos x}{(1-\sin x)(1 + \sin^{2} x)} \,dx $$
     
     \item Find the particular solution of the differential equation 
     $e^{x}\tan y dx + (2 - e^{x})\sec^{2} y dy = 0$, given that $y = \frac{\pi}{4}$ when $x=0$. 
     
     \begin{center}
     \textbf{OR}
     \end{center}
     
     Find the particular solution of the differential equation $\frac{dy}{dx} + 2y\tan x = \sin x$, given that $y = 0$ when $x = \frac{\pi}{3}$
     
     \item Let $\vec{a} = 4\hat{i} + 5\hat{j} - \hat{k}$, $\vec{b} = \hat{i} - 4\hat{j} + 5\hat{k}$ and $\vec{c} = 3\hat{i} + \hat{k} - \hat{k}$. Find a vector $\vec{d}$ which is perpendicular to both $\vec{c}$ and $\vec{b}$ and $\vec{d}$. $\vec{a} = 21$
     
     \item Find the shortest distance between the lines 

     $$\vec{r}  = (4\hat{i} - \hat{j}) + \lambda(\hat{i} + 2\hat{j} - 3\hat{k}) \hspace{2mm} and $$  $$\vec{r} = \hat{i} - \hat{j} + 2\hat{k} + \mu(2\hat{i} + 4\hat{j} - 5\hat{k})$$

     
     
     \item Suppose a girl throws a die. If she gets 1 or 2, she tosses a coin three times and notes the number of tails. If she gets 3, 4, 5 or 6, she tosses a coin once and notes whether a ‘head’ or ‘tail’ is obtained. If she obtained exactly one ‘tail’, what is the probability that she threw 3, 4, 5 or 6 with the die ?
    
    \item Two numbers are selected at random (without replacement) from the first five positive integers. Let X denote the larger of the two numbers obtained. Find the mean and variance of X.
    
    \begin{center}
    \section*{Section D}
    \end{center}
    
    \begin{flushleft}
    \textit{Question numbers 24 to 29 carry 6 marks each.}
    \end{flushleft}
    
    \item Let A = \{ x$\epsilon$ Z : 0$\leq$ x$\leq$12\}. Show that\\
    R = \{(a,b):a,b $\epsilon$ A,$\lvert$a-b$\rvert$ is divisible by 4\} is an equivalance class[2]
    
    \begin{center}
    \textbf{OR}
    \end{center}
    
    Show that the function f : R $\rightarrow$ R defined by $f(x) = \frac{x}{x^{2} + 1}$, $\forall$ x $\epsilon$ R is neither one-one nor onto. Also, if g : R $\rightarrow$ R is defined as g(x) = 2x - 1, find fog(x).
    
    \item\noindent If A = 
    $\begin{bmatrix}
     2 & -3 & 5 \\
     3 & 2 & -4 \\
     1 & 1 & -2 \\
     \end{bmatrix} $,
     find $A^{-1}$. Use it to solve the system of equations 
     $$2x - 3y + 5z = 11$$
     $$3x + 2y -4z = -5$$
     $$x + y – 2z = – 3.$$
     
     \begin{center}
     \textbf{OR}
     \end{center}
     
     Using elementary row transformations, find the inverse of the matrix
     $$\begin{bmatrix}
     1 & 2 & 3 \\
     2 & 5 & 7 \\
     -2 & -4 & -5 \\
     \end{bmatrix} $$
     
     \item Using integration, find the area of the region in the first quadrant enclosed by the x-axis, the line y = x and the circle $x^{2} + y^{2} = 32$.
     
     \item Evaluate :
     $$ \int_{0}^{\frac{\pi}{4}} \frac{\sin x + \cos x}{16 + 9\sin 2x} \,dx $$
     
     \begin{center}
     \textbf{OR}
     \end{center}
     
     Evaluate
     $$ \int_{1}^{2} (x^{2} + 3x + e^{x}) \,dx $$
     as the limit of the sum.
     
     \item Find the distance of the point (– 1, – 5, – 10) from the point of ntersection of the line $\vec{r} = 2\hat{i} - \hat{j} + 2\hat{k} +\lambda(3\hat{i} + 4\hat{j} + 2\hat{k})$ and the plane $\vec{r}$.$(\hat{i} - \hat{j} + \hat{k}) = 5$
     
     \item A factory manufactures two types of screws A and B, each type requiring the use of two machines, an automatic and a hand-operated. It takes 4 minutes on the automatic and 6 minutes on the hand-operated machines to manufacture a packet of screws ‘A’ while it takes 6 minutes
on the automatic and 3 minutes on the hand operated machine to manufacture a packet of screws ‘B’. Each machine is available for at most
4 hours on any day. The manufacturer can sell a packet of screws ‘A’ at a profit of 70 paise and screws ‘B’ at a profit of < 1. Assuming that he can sell all the screws he manufactures, how many packets of each type should the factory owner produce in a day in order to maximize his profit ? Formulate the above LPP and solve it graphically and find the maximum profit.
     
    
    
     
     
     
     \end{enumerate}

\end{document}


